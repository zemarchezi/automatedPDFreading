\documentclass[11pt, oneside]{article}
\usepackage[bottom=2.5cm,top=2.5cm]{geometry}
\geometry{a4paper}
\usepackage{graphicx}
\usepackage{amssymb}
\usepackage[utf8]{inputenc}
\usepackage[brazil]{babel}
\usepackage{color}
\usepackage{float}
\usepackage{hyperref}
\bibliographystyle{apalike}
\usepackage{indentfirst}

\title{Briefing Space Weather}
\date{2022/04/25}

\begin{document}
\maketitle 

 \section{Sun} 
 \subsection{Responsible: José Cecatto}

29/11 – Sem vento rápido; 3 CME podem ter uma componente para a Terra; (26) SB Prev cheg - 29/Nov, 06:14Z - 11:00Z; \\ 30/11 – Sem vento rápido; Sem CME dirigida para a Terra; Chegada de uma C.I.R. causou uma temp. geomagnética; \\ 01/12 – Vento rápido (< 600 km/s); 1 CME podem ter uma componente para a Terra; \\ 02/12 – Vento rápido (< 550 km/s); Sem CME dirigida para a Terra; (29) SB Prev cheg – 02/Dez., 05:52Z – 20:00Z; \\ 03/12 – Vento rápido (< 500 km/s); 2 CME podem ter uma componente para a Terra; \\ 04/12 – Vento rápido (< 550 km/s); 6 CME podem ter uma componente para a Terra – 1 halo parcial; \\ 05/12 – Vento rápido (< 500 km/s); 6 CME podem ter uma componente para a Terra – 1 halo parcial; Halo par., SB Prev \\ cheg – Dec/09, 18:00Z – 23:00Z; às 07:19 UT um flare M1.4 gerou um moderado blecaute radio em ondas curtas; \\ 06/12 – Vento rápido (=< 500 km/s); 1 CME podem ter uma componente para a Terra; Halo parc, SB Prev cheg – Dec/11, \\ 15:00Z; \\ Prev.: Vento rápido esperado p/ 07/Dez.; baixa probabilidade de “flares” (1\% M, 1\% X) nos próximos 02 dias; \\ eventualmente algum outro CME pode apresentar componente dirigida para a Terra.

\end{document}